\usepackage[scale=0.8,sfdefault,light]{roboto} 

\usetheme{Madrid}
\setbeamercovered{transparent}

\definecolor{uoa-blue}{RGB}{3,80,133}
\definecolor{uoa-light-blue}{RGB}{0,154,199}
\definecolor{uoa-grey}{RGB}{230,232,231}

\setbeamercolor{structure}{fg=uoa-blue}

\setbeamertemplate{blocks}[default]
\setbeamercolor{block title}{bg=uoa-blue,fg=white}
\setbeamercolor{block body}{bg=uoa-grey,fg=black}

\setbeamertemplate{title page}[default]

\setbeamertemplate{navigation symbols}{}
\setbeamertemplate{bibliography item}{\insertbiblabel}

\setbeamertemplate{itemize items}[default]
\setbeamertemplate{enumerate items}[default]

\setbeamertemplate{section in toc}[default]
\setbeamertemplate{subsection in toc}[default]

%\usepackage{fontspec}
%\defaultfontfeatures{Ligatures=TeX}
%\setromanfont{Verdana}


\setbeamercolor{normal text}{fg=black,bg=}
\setbeamercolor{alerted text}{fg=black,bg=}
\usebeamercolor{normal text}
\setbeamercovered{%
	again covered={\opaqueness<1->{15}}}

\newenvironment{myitemize}
{ \begin{itemize}
		\setlength{\itemsep}{6pt}     }
	{ \end{itemize}                  } 

\newenvironment{mypiecewiseitemize}
{ \begin{itemize}[<+>]
		\setlength{\itemsep}{6pt}     }
	{ \end{itemize}                  } 

\newenvironment{mycloseitemize}
{ \begin{itemize}
		\setlength{\itemsep}{3pt}     }
	{ \end{itemize}                  } 

\newenvironment<>{varblock}[2][\textwidth]{%
	\setlength{\textwidth}{#1}
	\begin{actionenv}#3%
		\def\insertblocktitle{#2}%
		\par%
		\usebeamertemplate{block begin}}
	{\par%
		\usebeamertemplate{block end}%
	\end{actionenv}}
	
	%\AtBeginSection[]
	%{
	%	\begin{frame}<beamer>
	%		\frametitle{Overview}
	%		\tableofcontents[currentsection]
	%	\end{frame}
	%
	
	\newenvironment{variableblock}[3]{%
		\setbeamercolor{block body}{#2}
		\setbeamercolor{block title}{#3}
		\begin{block}{#1}}{\end{block}}
	
	% Keys to support piece-wise uncovering of elements in TikZ pictures:
	% \node[visible on=<2->](foo){Foo}
	% \node[visible on=<{2,4}>](bar){Bar}   % put braces around comma expressions
	%
	% Internally works by setting opacity=0 when invisible, which has the 
	% adavantage (compared to \node<2->(foo){Foo} that the node is always there, hence
	% always consumes space plus that coordinate (foo) is always available.
	%
	% The actual command that implements the invisibility can be overriden
	% by altering the style invisible. For instance \tikzsset{invisible/.style={opacity=0.2}}
	% would dim the "invisible" parts. Alternatively, the color might be set to white, if the
	% output driver does not support transparencies (e.g., PS) 
	%
	\tikzset{
		invisible/.style={opacity=0},
		visible on/.style={alt={#1{}{invisible}}},
		alt/.code args={<#1>#2#3}{%
			\alt<#1>{\pgfkeysalso{#2}}{\pgfkeysalso{#3}} % \pgfkeysalso doesn't change the path
		},
	}
	
	\tikzset{onslide/.code args={<#1>#2}{%
			\only<#1>{\pgfkeysalso{#2}}
		}}
		\tikzstyle{highlight}=[red!90, fill=red!5]
		\tikzstyle{highlightsource}=[blue!90, fill=blue!5]
		\tikzstyle{borderhighlight}=[red!90, text=black, fill=red!5]
		\tikzstyle{texthighlight}=[text=red!90, fill=red!5]