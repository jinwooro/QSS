\documentclass[twoside,a4paper,12pt]{article}

\usepackage{tabularx}
\usepackage{amsmath}
\usepackage{amsthm}
\usepackage{graphicx}
\usepackage{subfigure}
\usepackage{bm}
\usepackage[format=plain,labelfont=up]{caption}
\usepackage{multirow}
\usepackage{afterpage}
\usepackage{booktabs} %for tables to use functions such as \toprule

\usepackage{amssymb}
\usepackage{mathrsfs}

\newcommand{\jin}[1]{{\textsf{\color{red} \small{[#1 -jin-]}}}}
\newcommand\ignore[1]{}
\usepackage[final]{changes}

\theoremstyle{definition}
\newtheorem{definition}{Definition}
\newtheorem{assumption}{Assumption}
\newtheorem{theorem}{Theorem}
\newtheorem{corollary}{Corollary}
\newtheorem{proposition}{Proposition}
\newtheorem{lemma}{Lemma}
\newtheorem{remark}{Remark}

\usepackage{hyperref}
\usepackage{bookmark}



\begin{document}
	
\begin{definition}
	A \emph{Hybrid Input Output Automaton (HIOA)} is a tuple:
	\begin{equation}
	H = \langle Loc, X, I, O, Init, f, h, Inv, E, G, R \rangle
	\end{equation}
	\begin{itemize}
		\item $Loc = \{ l_0, l_1, ... , l_n\}$ represents $n$ number of discrete
		control modes or locations with $l_0$ being the initial location.
		\item $X$ is a finite collection of continuous state variables, with its domain represented as $\mathbf{X} = \mathbb{R}^n$.
		\item $I = I_D \cup I_C$ is a finite collection of input variables, where $I_D$ is discrete inputs and $I_C$ is continuous inputs, with their domains $\mathbf{I}$, $\mathbf{I}_D$, and $\mathbf{I}_C$, respectively.
		\item $O = O_D \cup O_C$ is a finite collection of output variables, where
		$O_D$ is discrete outputs and $O_C$ is continuous outputs. Their domains are $\mathbf{O}$, $\mathbf{O}_D$, and $\mathbf{O}_C$.
		\item $Init \subseteq \{l_0\} \times \mathbf{X}$ is the system initialization. \jin{Should we not set the initial output values? because these values will be consumed in the dependent blocks.}
		\item $f:Loc \times \mathbf{X} \times \mathbf{I} \rightarrow \mathbf{X}$ is a vector field for specification of the ODEs. Function $f(l,x,i)$ is globally \emph{Lipschitz} continuous in $x \in \mathbf{X}$ and $i \in \mathbf{I}$.
		\item $h : Loc \times \mathbf{X} \rightarrow \mathbf{O}$ is a vector field. Function $h(i,x)$ is globally \emph{Lipschitz} continuous in $ x\in \mathbf{X}$.
		\item $Inv : Loc \rightarrow 2^{ \mathbf{X} \times \mathbf{I} }$ assigns to each $l \in Loc$ an invariant set. For example, $Inv(l)$ constrains the possible valuations for the continuous variables and input variables when the control of the hybrid system is in location $l$.
		\item $E \subseteq Loc \times Loc$ is a collection of discrete edges.
		\item $G : E \rightarrow 2^{ \mathbf{X} \times \mathbf{I} }$ assigns to each $e \in E$ a guard.
		\item $R : E \times \mathbf{X} \times \mathbf{I} \rightarrow 2^{\mathbf{X}}$ assigns to each $e = (l,l')\in E, x\in \mathbf{X}, i \in \mathbf{I}$ a reset relation.
	\end{itemize}
	\label{def:HIOA}
\end{definition}

\pagebreak


\begin{definition}
	A discrete hybrid timeset is a sequence of intervals $\tau = \{ \tau_0, \tau_1, \tau_2, ..., \tau_{k-1} \}$
	Consider a predefined time bound $T$ and a fixed length $\delta \in \mathbb{R}^{\geq0}$. Then, a discrete timeset is a collection of time instances , such that:
	\begin{enumerate}
		\item The number of time instances $k = \lfloor{\frac{T}{\delta}}\rfloor$, i.e., the cardinality of the set $| \tau | = k$.
		\item The value of each time instance $\tau_i = i \times \delta$. Hence, $\tau_i < \tau_j$ is always true if $i < j$.
	\end{enumerate}
	\label{ch1:def:timeset}
\end{definition}

\begin{definition}
	The progression of the $\tau$ value is a function $\Omega$ which is defined as:
	\begin{equation}
			\Omega : \tau \times ODEs \times Gs \rightarrow \tau
	\end{equation}
	$$ \tau_{n+1} = \Omega(\tau, ODEs, Gs) $$
\end{definition}


\end{document}

