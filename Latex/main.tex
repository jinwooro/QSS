\documentclass[twoside,a4paper,12pt]{article}

\usepackage{tabularx}
\usepackage{amsmath}
\usepackage{amsthm}
\usepackage{graphicx}
\usepackage{subfigure}
\usepackage{bm}
\usepackage[format=plain,labelfont=up]{caption}
\usepackage{multirow}
\usepackage{afterpage}
\usepackage{booktabs} %for tables to use functions such as \toprule

\usepackage{amssymb}
\usepackage{mathrsfs}

\newcommand{\jin}[1]{{\textsf{\color{red} \small{[#1 -jin-]}}}}
\newcommand\ignore[1]{}
\usepackage[final]{changes}

\theoremstyle{definition}
\newtheorem{definition}{Definition}
\newtheorem{assumption}{Assumption}
\newtheorem{theorem}{Theorem}
\newtheorem{corollary}{Corollary}
\newtheorem{proposition}{Proposition}
\newtheorem{lemma}{Lemma}
\newtheorem{remark}{Remark}

\usepackage{hyperref}
\usepackage{bookmark}



\begin{document}

\section{Paper Title}
A Quantized-state semantics for Simulink/Stateflow

\section{Introduction}
A system consisting of distributed controllers that interact with the physical environment is called Cyber-Physical Systems (CPS). 
Recently emerging advanced technologies in various applications including biomedical devices, Intelligent Transportation Systems, Industrial automation, and robotics are essentially CPSs as the robots, autonomous vehicles, smart devices, and wearable devices interact with the physical world.

These system requires stringent safety validation as the failure may results in catastrophic consequence.


\section{Problem Statement}
\begin{itemize}
	\item Problem statement
		\begin{itemize}
			\item Accurate level-crossing detection of HA simulation based on quantized state system semantics. 
			\item Model translation from Simulink to QSHA and performing QSS simulaion.
		\end{itemize}
	\item Running Example
	\item Quantized state system (QSS) syntax and semantics
	\item Single QSS execution vs. sequential QSS execution
	\item Simulink model translation
		\begin{itemize}
			\item Simulink model syntax (definition)
			\item Mapping function from Simulink to QSHA.
		\end{itemize}
	\item Simulation result
		\begin{itemize}
			\item comparison with the Simulink model
			\item Number of simulation steps
		\end{itemize}
\end{itemize}

\section{Simulink Stateflow}
Simulink Stateflow is the frontend structure in our study. Simulink can be defined as a tuple:

\begin{definition}
	A Simulink model $M = <Name,  Blocks, >$
	
\end{definition}


\section{Overview}
For modeling of Cyber-physical systems (CPSs), it is required to capture the continuous physical phenomena and the discrete computer processes. Many useful tools such as Simulink allow modeling of CPS.

\section{Structure definitions}
\subsection{Hybrid Input Output Automata}
In general, the hybrid behavior is captured using Hybrid Input Output Automata (HIOA).
\begin{definition}
	A \emph{Hybrid Input Output Automaton (HIOA)} is a tuple:
	\begin{equation}
	H = \langle Loc, X, I, O, Init, f, h, Inv, E, G, R \rangle
	\end{equation}
	\begin{itemize}
		\item $Loc = \{ l_0, l_1, ... , l_n\}$ represents $n$ number of discrete
		control modes or locations with $l_0$ being the initial location.
		\item $X$ is a finite collection of continuous state variables, with its domain represented as $\mathbf{X} = \mathbb{R}^n$.
		\item $I = I_D \cup I_C$ is a finite collection of input variables, where $I_D$ is discrete inputs and $I_C$ is continuous inputs, with their domains $\mathbf{I}$, $\mathbf{I}_D$, and $\mathbf{I}_C$, respectively.
		\item $O = O_D \cup O_C$ is a finite collection of output variables, where
		$O_D$ is discrete outputs and $O_C$ is continuous outputs. Their domains are $\mathbf{O}$, $\mathbf{O}_D$, and $\mathbf{O}_C$.
		\item $Init \subseteq \{l_0\} \times \mathbf{X}$ is the system initialization.
		\item $f:Loc \times \mathbf{X} \times \mathbf{I} \rightarrow \mathbf{X}$ is a vector field for specification of the ODEs. Function $f(l,x,i)$ is globally \emph{Lipschitz} continuous in $x \in \mathbf{X}$ and $i \in \mathbf{I}$.
		\item $h : Loc \times \mathbf{X} \rightarrow \mathbf{O}$ is a vector field. Function $h(i,x)$ is globally \emph{Lipschitz} continuous in $ x\in \mathbf{X}$.
		\item $Inv : Loc \rightarrow 2^{ \mathbf{X} \times \mathbf{I} }$ assigns to each $l \in Loc$ an invariant set. For example, $Inv(l)$ constrains the possible valuations for the continuous variables and input variables when the control of the hybrid system is in location $l$.
		\item $E \subseteq Loc \times Loc$ is a collection of discrete edges.
		\item $G : E \rightarrow 2^{ \mathbf{X} \times \mathbf{I} }$ assigns to each $e \in E$ a guard.
		\item $R : E \times \mathbf{X} \times \mathbf{I} \rightarrow 2^{\mathbf{X}}$ assigns to each $e = (l,l')\in E, x\in \mathbf{X}, i \in \mathbf{I}$ a reset relation.
	\end{itemize}
	\label{def:HIOA}
\end{definition}


\section{Solving Coupled Non-linear ODEs}
\subsection{One-dimensional Taylor series}
Taylor Series is defined as:
\begin{align}
& f(x)  = c_0 + c_1(x-a) + c_2(x-a)^2 + \dots + c_n(x-a)^n + \dots \\
\text{where, } & c_n  = \frac{f^n(a)}{n!} 
\end{align}
The general term $c_n$ is called Taylor Coefficient. The fully expressed Taylor series formula is:
\begin{equation}
	f(x) = f(a) + f'(a)(x-a) + \frac{f''(a)}{2!}(x-a)^2 + \dots + \frac{f^n(a)}{n!}(x-a)^n + \dots
\end{equation}
\subsection{Lagrange error bound}
The Lagrange error bound for the Taylor polynomial is the next term of Taylor series after the $x^n$ term:
\begin{equation}
	f(x) = f(a) + f'(a)(x-a) + \frac{f''(a)}{2!}(x-a)^2 + \dots + \frac{f^n(a)}{n!}(x-a)^n + R_n(x)
\end{equation}
where, the reminder term $R_n(x)$ is
\begin{equation}
	R_n(x) = \frac{f^{n+1}(c)(x-a)^{n+1}}{(n+1)!}
\end{equation}
The bounded error can be found as a constraint called the Lagrange error bound.
\begin{equation}
	R_n(x) < max \bigg| \frac{f^{n+1}(c)(x-a)^{n+1}}{(n+1)!} \bigg|
\end{equation}

\subsection{Multi-variable Taylor series}
Let $f$ be a function that is infinitely differentiable, and the initial values for two variables $(x,y) = (a,b)$.
\begin{equation}
	f(x,y) = f
\end{equation}


\subsection{Numerical differentiation}
Numerical differentiation is the process of finding the numerical value of a derivative of a given function at a given point.
\begin{equation}
f'(t) = \lim\limits_{h\rightarrow 0} \frac{f(t+h)-f(t)}{h}
\end{equation}
for some extremely small numerical value of $h \ll 1$.
The computational model of the numerical differentiation is as follow.
Let $\varepsilon \in \mathbb{R}$ be an extremely small value, $f(t)$ a function, and $t$ is the point we are finding the derivatives. Then,
\begin{align}
f'(t) &= \frac{f(t+\varepsilon)-f(t)}{\varepsilon}
\end{align}



Suppose we have two coupled non-linear ODEs where the independent variable is $t$.
\begin{gather}
	\dot{x}(t) = X(x,y,t) \\
	\dot{y}(t) = Y(x,y,t) 
\end{gather}
At $t=0$, the initial values are $x(0) = k_1$ and $y(0) = k_2$, where $k_1$ and $k_2$ are some constants.
Based on the initial values, we can calculate the first derivative of $x$ and $y$. $\dot{x}(0) = X(k_1, k_2, 0)$ and $\dot{y}(0) = Y(k_1, k_2, 0)$.
According to the Taylor series definition, we can approximate $x(t)$:
\begin{equation}
	x(t) = x(0) + x'(0) \cdot t + \frac{x''(0)}{2}\cdot t^2 + \frac{x'''(0)}{6}  \cdot t^3 + ... \frac{x^{(n)}(0)}{n!} \cdot t^n + ... 
\end{equation}
However, we only know $x(0)$ and $x'(0)$. It is straightforward to form the first order Taylor series, which is, in fact, identical to Euler's forward method.
\begin{equation}
	x(t) = x(0) + x'(0) \cdot t
\end{equation}
where the Lagrange error bound is $\frac{x''(0)}{2}\cdot t^2$ in this case.
However, the first order Taylor series often result in very small interval of convergence, hence we cannot estimate the value of $x$ with large progression of $t$.
To overcome, we need to know the value of the second, third, fourth derivatives to improve the approximation accuracy.









\begin{definition}
	The progression of the $\tau$ value is a function $\Omega$ which is defined as:
	\begin{equation}
			\Omega : \tau \times ODEs \times Gs \rightarrow \tau
	\end{equation}
	$$ \tau_{n+1} = \Omega(\tau, ODEs, Gs) $$
\end{definition}


\end{document}

