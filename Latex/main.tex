\documentclass[twoside,a4paper,12pt]{article}

\usepackage{tabularx}
\usepackage{amsmath}
\usepackage{amsthm}
\usepackage{graphicx}
\usepackage{subfigure}
\usepackage{bm}
\usepackage[format=plain,labelfont=up]{caption}
\usepackage{multirow}
\usepackage{afterpage}
\usepackage{booktabs} %for tables to use functions such as \toprule

\usepackage{amssymb}
\usepackage{mathrsfs}

\newcommand{\jin}[1]{{\textsf{\color{red} \small{[#1 -jin-]}}}}
\newcommand\ignore[1]{}
\usepackage[final]{changes}

\theoremstyle{definition}
\newtheorem{definition}{Definition}
\newtheorem{assumption}{Assumption}
\newtheorem{theorem}{Theorem}
\newtheorem{corollary}{Corollary}
\newtheorem{proposition}{Proposition}
\newtheorem{lemma}{Lemma}
\newtheorem{remark}{Remark}

\usepackage{hyperref}
\usepackage{bookmark}



\begin{document}

For modeling of Cyber-physical systems (CPSs), it is required to capture the continuous physical phenomena and the discrete computer processes. Many useful tools such as Simulink allow modeling of CPS.

\section{Structure definitions}
\subsection{Hybrid Input Output Automata}
In general, the hybrid behavior is captured using Hybrid Input Output Automata (HIOA).
\begin{definition}
	A \emph{Hybrid Input Output Automaton (HIOA)} is a tuple:
	\begin{equation}
	H = \langle Loc, X, I, O, Init, f, h, Inv, E, G, R \rangle
	\end{equation}
	\begin{itemize}
		\item $Loc = \{ l_0, l_1, ... , l_n\}$ represents $n$ number of discrete
		control modes or locations with $l_0$ being the initial location.
		\item $X$ is a finite collection of continuous state variables, with its domain represented as $\mathbf{X} = \mathbb{R}^n$.
		\item $I = I_D \cup I_C$ is a finite collection of input variables, where $I_D$ is discrete inputs and $I_C$ is continuous inputs, with their domains $\mathbf{I}$, $\mathbf{I}_D$, and $\mathbf{I}_C$, respectively.
		\item $O = O_D \cup O_C$ is a finite collection of output variables, where
		$O_D$ is discrete outputs and $O_C$ is continuous outputs. Their domains are $\mathbf{O}$, $\mathbf{O}_D$, and $\mathbf{O}_C$.
		\item $Init \subseteq \{l_0\} \times \mathbf{X}$ is the system initialization.
		\item $f:Loc \times \mathbf{X} \times \mathbf{I} \rightarrow \mathbf{X}$ is a vector field for specification of the ODEs. Function $f(l,x,i)$ is globally \emph{Lipschitz} continuous in $x \in \mathbf{X}$ and $i \in \mathbf{I}$.
		\item $h : Loc \times \mathbf{X} \rightarrow \mathbf{O}$ is a vector field. Function $h(i,x)$ is globally \emph{Lipschitz} continuous in $ x\in \mathbf{X}$.
		\item $Inv : Loc \rightarrow 2^{ \mathbf{X} \times \mathbf{I} }$ assigns to each $l \in Loc$ an invariant set. For example, $Inv(l)$ constrains the possible valuations for the continuous variables and input variables when the control of the hybrid system is in location $l$.
		\item $E \subseteq Loc \times Loc$ is a collection of discrete edges.
		\item $G : E \rightarrow 2^{ \mathbf{X} \times \mathbf{I} }$ assigns to each $e \in E$ a guard.
		\item $R : E \times \mathbf{X} \times \mathbf{I} \rightarrow 2^{\mathbf{X}}$ assigns to each $e = (l,l')\in E, x\in \mathbf{X}, i \in \mathbf{I}$ a reset relation.
	\end{itemize}
	\label{def:HIOA}
\end{definition}

\subsection{Block}
\begin{definition}
	A block is a tuple $B = <I, O, m>$, where $I$ is the set of input variables, $O$ is the output variables, $m : I \rightarrow O$ is the mapping function.
\end{definition}
The HIOA and Blocks are two types of objects in the system. 

\subsection{Discrete timeset}
Next, to describe the time progression of these blocks, we define a discrete timeset $\tau$.
\begin{definition}
	A discrete timeset is a sequence $\tau = \{ \tau_0, \tau_1, ..., \tau_{k-1}, \tau_{k} \}$, where $k \in \mathbb{N}$, $\tau_k \in \mathbb{R}$, $\tau_0 = 0$, and $\tau_{k-1} < \tau_{k}$.
	\label{ch1:def:timeset}
\end{definition}
In piecewise continuous functions, there may be jumps (i.e., discontinuity between subintervals). To capture these, we define a discrete left-sided limit timeset.
A left-sided limit timeset is a sequence $\tau^{-} = \{ \tau_1^{-}, \tau_2^{-}, ... , \tau_k^{-} \}$, such that for an arbitrary function $f(t)$, it satisfies:
\begin{equation}
f(\tau_i^{-}) = \lim\limits_{t\rightarrow\tau_i^{-}} f(t)
\end{equation}
Note that $\tau_0^-$ is missing in $\tau^-$ because $\tau$ starts from $\tau_0$. Since $\tau_0 = 0$ is the left end point (i.e., time cannot be smaller than zero), we do not consider the left-sided limit of $t=0$.

The general idea of discrete timeset is that the system progresses the time in the order $\tau_0, \tau_1^-, \tau_1, \tau_2^-, \tau_2, ... , \tau_n^-, \tau_n$. For all $i$, the time progresses between $\tau_i$ and $\tau_{i+1}^-$, and the continuous variables are updated. We enforce that during this time interval, there is no change in location such that the location at $\tau_{i+1}^-$ is identical to the location at $\tau_i$ for all HIOAs in the system.
Consequently, the ODEs active during this time interval are fixed and can be known.
On the other hand, the progress from $\tau_{i}^-$ to $\tau_{i}$ takes zero time, which is useful to capture the instantaneous transition between the locations.
Since the time is not passed, the continuous variables are not changed. 
However, the continuous variables are assigned to new values through the reset relation on the edge transition if any edge is enabled by the guard condition.
Since not all HIOAs enable the edge simultaneously, only some blocks may change the location.

\subsection{Overall system structure}
Consider two HIOA instances $H_1$ and $H_2$, and let $I^{H_1}$ and $I^{H_2}$ are their input variable sets, respectively.
Then, we denote $I^{H_1}_k$ to indicate a specific input $i_k \in I^{H_1}$. Similarly, $I^{H_2}_k$ indicates the input $i_k \in I^{H_2}$ in $H_2$.
Furthermore, we use the same notation for other components of HIOA. 
For example, $O^{H_1}_k$ and $X^{H_1}_k$.
The same indication applies to the Blocks, so that $I^{B_1}_k$ indicates the input variable $i_k$ in $B_1$.
Now, we define the system.
\begin{definition}
	A system $S = <\tau, \mathcal{H}, \mathcal{B}, \mathcal{I}, \mathcal{O}, \mathcal{L}, \Omega, \Psi>$
	\begin{itemize}
		\item $\tau$ is the discrete timeset.
		\item $\mathcal{H}$ is the set of HIOA.
		\item $\mathcal{B}$ is the set of Blocks.
		\item $\mathcal{I} = \bigcup\limits_{Z\in\mathcal{H} \cup \mathcal{B}}I^Z$ is the collection of all input variables of HIOAs and Blocks in the system.
		\item $\mathcal{O} = \bigcup\limits_{Z\in\mathcal{H} \cup \mathcal{B}}O^Z$ is the collection of all output variables of HIOAs and Blocks in the system.
		\item $\mathcal{L} : \mathcal{I} \rightarrow \mathcal{O}$ is the I/O relation. 
		\item $\Omega(\tau_i, ODE(\tau_i), CN(\tau_i)) = \tau_{i+1}$ is the function that computes the next discrete time.
		\item $\Psi(\tau_i^-, \mathcal{H}, \mathcal{B}) $
	\end{itemize}
\end{definition}

\section{Execution semantics}
The execution of a single HIOA is defined as follows.
\begin{definition}
Let $\tau$ be the discrete timeset, $l : \tau \rightarrow Loc$, $x : \tau \rightarrow X$, $i : \tau \rightarrow I$, $o: \tau \rightarrow O$. Then, a discrete execution of a single HIOA satisfies:
	\begin{enumerate}
		\item Initialization: $(l(\tau_0), x(\tau_0)) \in Init$
		\item Invariant: $\forall \tau_i \in \tau$, $(x(\tau_i), i(\tau_i)) \in Inv(l(\tau_i))$, and $(x(\tau_{i+1}^-), i(\tau_{i+1}^-)) \in Inv(l(\tau_i))$. However, $(x(\tau_i^-), i(\tau_i^-)) \notin Inv(l(\tau_i))$ is allowed to capture the discontinuity between two consecutive discrete times.
		\item Continuous variable update: for all times, $l(\tau_i) = l(\tau_{i+1}^-)$ is satisfied. The vector field $f$ is approximated by Taylor polynomial $F$ over the time interval $t \in [\tau_i, \tau_{i+1}^-]$, thus $x(t) = F(t)$. The radius of convergence $r$ must satisfy $r \geq \tau_{i+1} - \tau_i$ to ensure the approximation $F$ is valid.
		\item Edge transition: $\forall \tau_i \in \tau$, $e = (l(\tau_{i+1}^-), l(\tau_{i+1})) \in E$, $(x(\tau_{i+1}^-), i(\tau_{i+1}^-))\in G(e)$, and $x(\tau_{i+1}) \in R(e, x(\tau_{i+1}^-), i(\tau_{i+1}^-))$
		\item Output update: $\forall t \in \tau$, $o(t) = h( l(t), x(t) )$
	\end{enumerate}
\end{definition}

On the other hand, the Blocks are executed differently depending on the Block types (e.g., arithmetic operations, integration, differentiation, etc...).

\subsection{Integrator Block}
\subsubsection{Continuous Integrator}
An input $i_1 = [c_0, c_1, c_2, ... , c_n ] \in I$ is a fixed size vector, where each value is a real number representing the Taylor series coefficients, such that it implies a function
\begin{equation}
	 f(t) = c_0 + c_1 \cdot t + c_2 \cdot t^2 + ... c_n \cdot t^n
\end{equation}
The other input $i_2 = \Delta t$ which is the time elapsed from the previous computation.
The output is the integration of the Taylor series over the time elapsed. Hence:
\begin{align}
	O(t) &= \int_{0}^{\Delta t} f(t) dt \\
	&= \int_{0}^{\Delta t} \bigg( c_0 + c_1 \cdot t + c_2 \cdot t^2 + ... c_n \cdot t^n \bigg) dt \\
	&= c_0 \cdot t + \frac{c_1}{2} \cdot t^2 + \frac{c_2}{3} \cdot t^3 ... \frac{c_n}{n+1} \cdot t^{n+1} \bigg|_0^{\Delta t}
\end{align}
The function $m:  \int_{0}^{\Delta t} c_0 + c_1 \cdot t + c_2 \cdot t^2 + ... c_n \cdot t^n dt$


\section{Solving Coupled Non-linear ODEs}
\subsection{One-dimensional Taylor series}
Taylor Series is defined as:
\begin{align}
& f(x)  = c_0 + c_1(x-a) + c_2(x-a)^2 + \dots + c_n(x-a)^n + \dots \\
\text{where, } & c_n  = \frac{f^n(a)}{n!} 
\end{align}
The general term $c_n$ is called Taylor Coefficient. The fully expressed Taylor series formula is:
\begin{equation}
	f(x) = f(a) + f'(a)(x-a) + \frac{f''(a)}{2!}(x-a)^2 + \dots + \frac{f^n(a)}{n!}(x-a)^n + \dots
\end{equation}
\subsection{Lagrange error bound}
The Lagrange error bound for the Taylor polynomial is the next term of Taylor series after the $x^n$ term:
\begin{equation}
	f(x) = f(a) + f'(a)(x-a) + \frac{f''(a)}{2!}(x-a)^2 + \dots + \frac{f^n(a)}{n!}(x-a)^n + R_n(x)
\end{equation}
where, the reminder term $R_n(x)$ is
\begin{equation}
	R_n(x) = \frac{f^{n+1}(c)(x-a)^{n+1}}{(n+1)!}
\end{equation}
The bounded error can be found as a constraint called the Lagrange error bound.
\begin{equation}
	R_n(x) < max \bigg| \frac{f^{n+1}(c)(x-a)^{n+1}}{(n+1)!} \bigg|
\end{equation}

\subsection{Multi-variable Taylor series}
Let $f$ be a function that is infinitely differentiable, and the initial values for two variables $(x,y) = (a,b)$.
\begin{equation}
	f(x,y) = f
\end{equation}


\subsection{Numerical differentiation}
Numerical differentiation is the process of finding the numerical value of a derivative of a given function at a given point.
\begin{equation}
f'(t) = \lim\limits_{h\rightarrow 0} \frac{f(t+h)-f(t)}{h}
\end{equation}
for some extremely small numerical value of $h \ll 1$.
The computational model of the numerical differentiation is as follow.
Let $\varepsilon \in \mathbb{R}$ be an extremely small value, $f(t)$ a function, and $t$ is the point we are finding the derivatives. Then,
\begin{align}
f'(t) &= \frac{f(t+\varepsilon)-f(t)}{\varepsilon}
\end{align}



Suppose we have two coupled non-linear ODEs where the independent variable is $t$.
\begin{gather}
	\dot{x}(t) = X(x,y,t) \\
	\dot{y}(t) = Y(x,y,t) 
\end{gather}
At $t=0$, the initial values are $x(0) = k_1$ and $y(0) = k_2$, where $k_1$ and $k_2$ are some constants.
Based on the initial values, we can calculate the first derivative of $x$ and $y$. $\dot{x}(0) = X(k_1, k_2, 0)$ and $\dot{y}(0) = Y(k_1, k_2, 0)$.
According to the Taylor series definition, we can approximate $x(t)$:
\begin{equation}
	x(t) = x(0) + x'(0) \cdot t + \frac{x''(0)}{2}\cdot t^2 + \frac{x'''(0)}{6}  \cdot t^3 + ... \frac{x^{(n)}(0)}{n!} \cdot t^n + ... 
\end{equation}
However, we only know $x(0)$ and $x'(0)$. It is straightforward to form the first order Taylor series, which is, in fact, identical to Euler's forward method.
\begin{equation}
	x(t) = x(0) + x'(0) \cdot t
\end{equation}
where the Lagrange error bound is $\frac{x''(0)}{2}\cdot t^2$ in this case.
However, the first order Taylor series often result in very small interval of convergence, hence we cannot estimate the value of $x$ with large progression of $t$.
To overcome, we need to know the value of the second, third, fourth derivatives to improve the approximation accuracy.









\begin{definition}
	The progression of the $\tau$ value is a function $\Omega$ which is defined as:
	\begin{equation}
			\Omega : \tau \times ODEs \times Gs \rightarrow \tau
	\end{equation}
	$$ \tau_{n+1} = \Omega(\tau, ODEs, Gs) $$
\end{definition}


\end{document}

